\chapter{付録  電流計測プログラム}
\begin{lstlisting}[basicstyle=\ttfamily\footnotesize]






//電流計測
\#include<FlexiTimer2.h>
\#include"Motor.h"

unsigned long timer_a = 0;
const int ts = 1;

const int pinI = 1;
int current = 0.0;
double var1 = 1;//1.1 / 1024 * 1.0 / 0.13;

int pwm = 0;
double voltage = 0.0;
double var2 = 12.0 / 255.0;

//モーター(入力1,2、PWM、加速度、サンプリング時間)、上限lim、下限lim
//LimitMotor lift(new S_Motor(35, 34, 4, 255, ts), 66, 67);//昇降モーター
//Motor lift(2, 3, 5);

//電流計


void setup() {
    Serial.begin(115200);
    //lift.is_brake = false;
    analogReference(INTERNAL);
    FlexiTimer2::set(ts, motorCalculat);//サンプリング時間毎にモーターの速度計算を行うタイマー割り込みの設定
    FlexiTimer2::start();//タイマー割り込みの開始
    current = var1 * analogRead(pinI);
    voltage = var2 * pwm;
    //pwm = 0;
    pinMode(2, OUTPUT);
    pinMode(3, OUTPUT);
    pinMode(5, OUTPUT);
}

void loop() {
    current = var1 * analogRead(pinI);
    //lift.set_rotation_velocity(pwm);
    digitalWrite(3, HIGH);
    digitalWrite(2, LOW);
    analogWrite(5, 0);
    //pwm = (timer_a/1000.0 * 1.2 * 255.0/12.0);
}

void motorCalculat() {
//    Serial.print(timer_a);
//    Serial.print(' ');
    Serial.print(pwm);
    Serial.print(' ');
    Serial.println(current);
    //lift.loop();
    timer_a += ts;
}

 \end{lstlisting}
